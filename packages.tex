\usepackage[T1]{fontenc}
\usepackage[utf8]{inputenc}

\usepackage{microtype}

\usepackage[numbers]{natbib}

\usepackage{soul} % Underline, Strikethrough
\usepackage{xcolor}
\usepackage{xurl, hyperref}
\usepackage{tcolorbox}
\usepackage[super]{nth}
\usepackage[ruled]{algorithm2e}
\usepackage[shortlabels, inline]{enumitem}
% Tables and Figures
\usepackage{float, graphicx, wrapfig, subcaption}
\usepackage{booktabs, tabularx, multirow, tablefootnote}
% Monospaced Code Blocks
\usepackage{fancyvrb, listings}
% Math Packages
\usepackage{nicefrac}
\usepackage{mathrsfs}
\usepackage{mathtools}
\usepackage{amsmath, amsthm, amssymb, amsbsy}

\lstset{ %
  backgroundcolor=\color{white},   % choose the background color; you must add \usepackage{color} or \usepackage{xcolor}; should come as last argument
  basicstyle=\ttfamily,            % the size of the fonts that are used for the code
  breakatwhitespace=false,         % sets if automatic breaks should only happen at whitespace
  breaklines=true,                 % sets automatic line breaking
  captionpos=b,                    % sets the caption-position to bottom
  commentstyle=\color{gray},       % comment style
  % deletekeywords={...},            % if you want to delete keywords from the given language
  escapeinside={\%*}{*)},          % if you want to add LaTeX within your code
  extendedchars=true,              % lets you use non-ASCII characters; for 8-bits encodings only, does not work with UTF-8
  % frame=single,                    % adds a frame around the code
  keepspaces=true,                 % keeps spaces in text, useful for keeping indentation of code (possibly needs columns=flexible)
  keywordstyle=\color{blue},       % keyword style
  language=C++,                    % the language of the code
  % morekeywords={*,...},            % if you want to add more keywords to the set
  numbers=none,                    % where to put the line-numbers; possible values are (none, left, right)
  numbersep=5pt,                   % how far the line-numbers are from the code
  numberstyle=\color{black},       % the style that is used for the line-numbers
  rulecolor=\color{black},         % if not set, the frame-color may be changed on line-breaks within not-black text (e.g. comments (green here))
  showspaces=false,                % show spaces everywhere adding particular underscores; it overrides 'showstringspaces'
  showstringspaces=false,          % underline spaces within strings only
  showtabs=false,                  % show tabs within strings adding particular underscores
  stepnumber=1,                    % the step between two line-numbers. If it's 1, each line will be numbered
  stringstyle=\color{red},         % string literal style
  tabsize=4,                       % sets default tabsize to 4 spaces
  % title=\lstname                   % show the filename of files included with \lstinputlisting; also try caption instead of title
}

\newcommand{\valignc}[1]{\raisebox{-0.5\height}{#1}}
\newcommand{\Emph}[1]{\ul{\textbf{#1}}}
\newcommand{\EmphMathText}[1]{\underline{\mathbf{#1}}}
\newcommand{\EmphMath}[1]{\underline{\boldsymbol{#1}}}

\newtheorem{theorem}{Theorem}
\newtheorem{claim}[theorem]{Claim}
\newtheorem{lemma}[theorem]{Lemma}
\newtheorem{corollary}[theorem]{Corollary}
\newtheorem{conjecture}[theorem]{Conjecture}
\newtheorem{proposition}[theorem]{Proposition}
\newtheorem*{remark}{Remark}
\newtheorem*{example}{Example}
\newtheorem*{observation}{Observation}
